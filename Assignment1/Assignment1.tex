\documentclass[journal,12pt,twocolumn]{IEEEtran}

\usepackage{setspace}
\usepackage{gensymb}
\singlespacing
\usepackage[cmex10]{amsmath}

\usepackage{amsthm}

\usepackage{mathrsfs}
\usepackage{txfonts}
\usepackage{stfloats}
\usepackage{bm}
\usepackage{cite}
\usepackage{cases}
\usepackage{subfig}

\usepackage{longtable}
\usepackage{multirow}

\usepackage{enumitem}
\usepackage{mathtools}
\usepackage{steinmetz}
\usepackage{tikz}
\usepackage{circuitikz}
\usepackage{verbatim}
\usepackage{tfrupee}
\usepackage[breaklinks=true]{hyperref}
\usepackage{graphicx}
\usepackage{tkz-euclide}

\usetikzlibrary{calc,math}
\usepackage{listings}
    \usepackage{color}                                            %%
    \usepackage{array}                                            %%
    \usepackage{longtable}                                        %%
    \usepackage{calc}                                             %%
    \usepackage{multirow}                                         %%
    \usepackage{hhline}                                           %%
    \usepackage{ifthen}                                           %%
    \usepackage{lscape}     
\usepackage{multicol}
\usepackage{chngcntr}

\DeclareMathOperator*{\Res}{Res}

\renewcommand\thesection{\arabic{section}}
\renewcommand\thesubsection{\thesection.\arabic{subsection}}
\renewcommand\thesubsubsection{\thesubsection.\arabic{subsubsection}}

\renewcommand\thesectiondis{\arabic{section}}
\renewcommand\thesubsectiondis{\thesectiondis.\arabic{subsection}}
\renewcommand\thesubsubsectiondis{\thesubsectiondis.\arabic{subsubsection}}


\hyphenation{op-tical net-works semi-conduc-tor}
\def\inputGnumericTable{}                                 %%

\lstset{
%language=C,
frame=single, 
breaklines=true,
columns=fullflexible
}
\begin{document}


\newtheorem{theorem}{Theorem}[section]
\newtheorem{problem}{Problem}
\newtheorem{proposition}{Proposition}[section]
\newtheorem{lemma}{Lemma}[section]
\newtheorem{corollary}[theorem]{Corollary}
\newtheorem{example}{Example}[section]
\newtheorem{definition}[problem]{Definition}

\newcommand{\BEQA}{\begin{eqnarray}}
\newcommand{\EEQA}{\end{eqnarray}}
\newcommand{\define}{\stackrel{\triangle}{=}}
\bibliographystyle{IEEEtran}
\raggedbottom
\setlength{\parindent}{0pt}
\providecommand{\mbf}{\mathbf}
\providecommand{\pr}[1]{\ensuremath{\Pr\left(#1\right)}}
\providecommand{\qfunc}[1]{\ensuremath{Q\left(#1\right)}}
\providecommand{\sbrak}[1]{\ensuremath{{}\left[#1\right]}}
\providecommand{\lsbrak}[1]{\ensuremath{{}\left[#1\right.}}
\providecommand{\rsbrak}[1]{\ensuremath{{}\left.#1\right]}}
\providecommand{\brak}[1]{\ensuremath{\left(#1\right)}}
\providecommand{\lbrak}[1]{\ensuremath{\left(#1\right.}}
\providecommand{\rbrak}[1]{\ensuremath{\left.#1\right)}}
\providecommand{\cbrak}[1]{\ensuremath{\left\{#1\right\}}}
\providecommand{\lcbrak}[1]{\ensuremath{\left\{#1\right.}}
\providecommand{\rcbrak}[1]{\ensuremath{\left.#1\right\}}}
\theoremstyle{remark}
\newtheorem{rem}{Remark}
\newcommand{\sgn}{\mathop{\mathrm{sgn}}}
\providecommand{\abs}[1]{\vert#1\vert}
\providecommand{\res}[1]{\Res\displaylimits_{#1}} 
\providecommand{\norm}[1]{\lVert#1\rVert}
%\providecommand{\norm}[1]{\lVert#1\rVert}
\providecommand{\mtx}[1]{\mathbf{#1}}
\providecommand{\mean}[1]{E[ #1 ]}
\providecommand{\fourier}{\overset{\mathcal{F}}{ \rightleftharpoons}}
%\providecommand{\hilbert}{\overset{\mathcal{H}}{ \rightleftharpoons}}
\providecommand{\system}{\overset{\mathcal{H}}{ \longleftrightarrow}}
	%\newcommand{\solution}[2]{\textbf{Solution:}{#1}}
\newcommand{\solution}{\noindent \textbf{Solution: }}
\newcommand{\cosec}{\,\text{cosec}\,}
\providecommand{\dec}[2]{\ensuremath{\overset{#1}{\underset{#2}{\gtrless}}}}
\newcommand{\myvec}[1]{\ensuremath{\begin{pmatrix}#1\end{pmatrix}}}
\newcommand{\mydet}[1]{\ensuremath{\begin{vmatrix}#1\end{vmatrix}}}
\numberwithin{equation}{subsection}
\makeatletter
\@addtoreset{figure}{problem}
\makeatother
\let\StandardTheFigure\thefigure
\let\vec\mathbf
\renewcommand{\thefigure}{\theproblem}
\def\putbox#1#2#3{\makebox[0in][l]{\makebox[#1][l]{}\raisebox{\baselineskip}[0in][0in]{\raisebox{#2}[0in][0in]{#3}}}}
     \def\rightbox#1{\makebox[0in][r]{#1}}
     \def\centbox#1{\makebox[0in]{#1}}
     \def\topbox#1{\raisebox{-\baselineskip}[0in][0in]{#1}}
     \def\midbox#1{\raisebox{-0.5\baselineskip}[0in][0in]{#1}}
\vspace{3cm}
\title{Assignment 1}
\author{Raghav Juyal - EP20BTECH11018}
\maketitle
\newpage
\bigskip
\renewcommand{\thefigure}{\theenumi}
\renewcommand{\thetable}{\theenumi}
Download all python codes from 
\begin{lstlisting}
https://github.com/RaghavJuyal/AI1103/blob/main/Assignment1/Codes/Assignment1.py
\end{lstlisting}
%
and latex-tikz codes from 
%
\begin{lstlisting}
https://github.com/RaghavJuyal/AI1103/tree/main/Assignment1/Assignment1.tex
\end{lstlisting}
\section*{Question 5.18}
An insurance company insured 2000 scooter drivers, 4000 car drivers and 6000 truck drivers. The probability of an accident are 0.01, 0.03 and 0.15 respectively. One of the insured persons meets with an accident. What is the probability that he is a scooter driver?

\section*{Solution}
The total number of drivers is 12000 (2000+4000+6000).\\
\\ Let $X \in \{0,1,2\}$ represent a random variable where 0 represents scooter driver, 1 represents car driver and 2 represents truck driver.
\\ Let $Y \in \{0,1\}$ represent a random variable where 0 represents no accidents and 1 represents an accident.\\
We can calculate:\\
\begin{align}
    Pr(X=0) = \frac{2000}{12000} = \frac{2}{12}\\
    Pr(X=1) = \frac{4000}{12000} = \frac{4}{12}\\
    Pr(X=2) = \frac{6000}{12000} = \frac{6}{12}
\end{align}

Also given: \\
\begin{align}
    Pr(Y=1|X=0) = 0.01 \\
    Pr(Y=1|X=1) = 0.03 \\
    Pr(Y=1|X=2) = 0.15
\end{align}
From definition we have,
\begin{align}
    Pr(E|F) = \frac{Pr(EF)}{Pr(F)}\\
    \implies Pr(EF) = Pr(E|F)\: Pr(F) \label{eq1}
\end{align}
From total probability theorem we get,
\begin{align}
    Pr(F) = \sum_{i=1}^n Pr(F|E_{i})\: Pr(E_{i}) \label{eq2}
\end{align}
From \ref{eq1} and \ref{eq2} we get,
\begin{align}
    Pr(F|E_{k}) = \frac{Pr(E_{k})}{\sum_{i=1}^n Pr(F|E_{i})\: Pr(E_{i})} \label{eq3}\\
    \nonumber\\
    \text{where }  1\leq k\leq n \nonumber
\end{align}
Here \ref{eq3} is Bayes' theorem.\\ \\
In the given question we require,
\begin{align}
    Pr(X=0|Y=1) \label{eq4}
\end{align}
\\Applying Bayes' theorem on \ref{eq4} we get,

\begin{align}
    Pr(X=0|Y=1) = \frac{Pr(Y=1|X=0)\: Pr(X=0)}{\sum_{i=0}^2 Pr(Y=1|X=i)\: Pr(X=i)}\label{eq5}
\end{align}

Replacing the values in \ref{eq5} we get,
\begin{align}
    \frac{0.01\times\frac{2}{12}}{0.01\times \frac{2}{12}+0.03\times \frac{4}{12}+0.15\times \frac{6}{12}}\\
    \nonumber\\
    \therefore \text{Answer }= \:\frac{1}{52}
\end{align}
\end{document}