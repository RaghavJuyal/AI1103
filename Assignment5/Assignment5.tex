\documentclass[journal,12pt,twocolumn]{IEEEtran}

\usepackage{setspace}
\usepackage{gensymb}
\singlespacing
\usepackage[cmex10]{amsmath}

\usepackage{amsthm}

\usepackage{mathrsfs}
\usepackage{txfonts}
\usepackage{stfloats}
\usepackage{bm}
\usepackage{cite}
\usepackage{cases}
\usepackage{subfig}

\usepackage{longtable}
\usepackage{multirow}

\usepackage{enumitem}
\usepackage{mathtools}
\usepackage{steinmetz}
\usepackage{tikz}
\usepackage{circuitikz}
\usepackage{verbatim}
\usepackage{tfrupee}
\usepackage[breaklinks=true]{hyperref}
\usepackage{graphicx}
\usepackage{tkz-euclide}


\usetikzlibrary{calc,math}
\usepackage{listings}
    \usepackage{color}                                            %%
    \usepackage{array}                                            %%
    \usepackage{longtable}                                        %%
    \usepackage{calc}                                             %%
    \usepackage{multirow}                                         %%
    \usepackage{hhline}                                           %%
    \usepackage{ifthen}                                           %%
    \usepackage{lscape}     
\usepackage{multicol}
\usepackage{chngcntr}

\DeclareMathOperator*{\Res}{Res}

\renewcommand\thesection{\arabic{section}}
\renewcommand\thesubsection{\thesection.\arabic{subsection}}
\renewcommand\thesubsubsection{\thesubsection.\arabic{subsubsection}}

\renewcommand\thesectiondis{\arabic{section}}
\renewcommand\thesubsectiondis{\thesectiondis.\arabic{subsection}}
\renewcommand\thesubsubsectiondis{\thesubsectiondis.\arabic{subsubsection}}



\hyphenation{op-tical net-works semi-conduc-tor}
\def\inputGnumericTable{}                                 %%

\lstset{
%language=C,
frame=single, 
breaklines=true,
columns=fullflexible
}
\begin{document}


\newtheorem{theorem}{Theorem}[section]
\newtheorem{problem}{Problem}
\newtheorem{proposition}{Proposition}[section]
\newtheorem{lemma}{Lemma}[section]
\newtheorem{corollary}[theorem]{Corollary}
\newtheorem{example}{Example}[section]
\newtheorem{definition}[problem]{Definition}

\newcommand{\BEQA}{\begin{eqnarray}}
\newcommand{\EEQA}{\end{eqnarray}}
\newcommand{\define}{\stackrel{\triangle}{=}}
\bibliographystyle{IEEEtran}
\raggedbottom
\setlength{\parindent}{0pt}
\providecommand{\mbf}{\mathbf}
\providecommand{\pr}[1]{\ensuremath{\Pr\left(#1\right)}}
\providecommand{\qfunc}[1]{\ensuremath{Q\left(#1\right)}}
\providecommand{\sbrak}[1]{\ensuremath{{}\left[#1\right]}}
\providecommand{\lsbrak}[1]{\ensuremath{{}\left[#1\right.}}
\providecommand{\rsbrak}[1]{\ensuremath{{}\left.#1\right]}}
\providecommand{\brak}[1]{\ensuremath{\left(#1\right)}}
\providecommand{\lbrak}[1]{\ensuremath{\left(#1\right.}}
\providecommand{\rbrak}[1]{\ensuremath{\left.#1\right)}}
\providecommand{\cbrak}[1]{\ensuremath{\left\{#1\right\}}}
\providecommand{\lcbrak}[1]{\ensuremath{\left\{#1\right.}}
\providecommand{\rcbrak}[1]{\ensuremath{\left.#1\right\}}}
\theoremstyle{remark}
\newtheorem{rem}{Remark}
\newcommand{\sgn}{\mathop{\mathrm{sgn}}}
\providecommand{\abs}[1]{\vert#1\vert}
\providecommand{\res}[1]{\Res\displaylimits_{#1}} 
\providecommand{\norm}[1]{\lVert#1\rVert}
%\providecommand{\norm}[1]{\lVert#1\rVert}
\providecommand{\mtx}[1]{\mathbf{#1}}
\providecommand{\mean}[1]{E[ #1 ]}
\providecommand{\fourier}{\overset{\mathcal{F}}{ \rightleftharpoons}}
%\providecommand{\hilbert}{\overset{\mathcal{H}}{ \rightleftharpoons}}
\providecommand{\system}{\overset{\mathcal{H}}{ \longleftrightarrow}}
	%\newcommand{\solution}[2]{\textbf{Solution:}{#1}}
\newcommand{\solution}{\noindent \textbf{Solution: }}
\newcommand{\cosec}{\,\text{cosec}\,}
\providecommand{\dec}[2]{\ensuremath{\overset{#1}{\underset{#2}{\gtrless}}}}
\newcommand*{\permcomb}[4][0mu]{{{}^{#3}\mkern#1#2_{#4}}}
\newcommand*{\perm}[1][-3mu]{\permcomb[#1]{P}}
\newcommand*{\comb}[1][-1mu]{\permcomb[#1]{C}}
\newcommand{\myvec}[1]{\ensuremath{\begin{pmatrix}#1\end{pmatrix}}}
\newcommand{\mydet}[1]{\ensuremath{\begin{vmatrix}#1\end{vmatrix}}}
\numberwithin{equation}{subsection}
\makeatletter
\@addtoreset{figure}{problem}
\makeatother
\let\StandardTheFigure\thefigure
\let\vec\mathbf
\renewcommand{\thefigure}{\theproblem}
\def\putbox#1#2#3{\makebox[0in][l]{\makebox[#1][l]{}\raisebox{\baselineskip}[0in][0in]{\raisebox{#2}[0in][0in]{#3}}}}
     \def\rightbox#1{\makebox[0in][r]{#1}}
     \def\centbox#1{\makebox[0in]{#1}}
     \def\topbox#1{\raisebox{-\baselineskip}[0in][0in]{#1}}
     \def\midbox#1{\raisebox{-0.5\baselineskip}[0in][0in]{#1}}
\vspace{3cm}
\title{Assignment 5}
\author{Raghav Juyal - EP20BTECH11018}
\maketitle
\newpage
\bigskip
\renewcommand{\thefigure}{\theenumi}
\renewcommand{\thetable}{\theenumi}
Download latex-tikz codes from 
%
\begin{lstlisting}
https://github.com/RaghavJuyal/AI1103/tree/main/Assignment5/Assignment5.tex
\end{lstlisting}
\section*{Question 113, CSIR UGC NET EXAM (Dec 2014)}
Let $X_1,X_2,...,X_n$ be independent and identically distributed Bernoulli($\theta$), where $0<\theta<1$ and $n>1$. Let the prior density of $\theta$ be proportional to $\frac{1}{\sqrt{\theta\,(1-\theta)}}$, $0<\theta<1$. Define $S=\sum_{i=1}^nXi$.\\[1pt] Then valid statements among the following are:
\begin{enumerate}[label = \arabic*.]
    \item The posterior mean of $\theta$ does not exist;
    \item The posterior mean of $\theta$ exists;
    \item The posterior mean of $\theta$ exists and it is larger than the maximum likelihood estimator for all values of S.
    \item The posterior mean of $\theta$ exists and it is larger than the maximum likelihood estimator for some values of S.
\end{enumerate}
\section*{Solution}
Let $f_\Theta(\theta)$ be the prior density and $f_{X|\Theta}(x|\theta)$ be the likelihood function.
\begin{align}
    f_\Theta(\theta) \propto \frac{1}{\sqrt{\theta\,(1-\theta)}}
\end{align}
\begin{align}
    f_{X|\Theta}(x|\theta) &= \prod_{i=1}^n\theta^{X_i}\,(1-\theta)^{1-X_i}\nonumber\\ 
    &=\theta^S\,(1-\theta)^{n-S}
\end{align}
Let MLE be the maximum likelihood estimator.
\begin{align}
    \ln{f_{X|\Theta}(x|\theta)} &= S\ln{\theta} + (n-S)\ln{(1-\theta)}\\
    \frac{\partial \ln{f_{X|\Theta}(x|\theta)}}{\partial \theta} &= \frac{S}{\theta} + \frac{S-n}{1-\theta} = 0 \nonumber\\
     \therefore \text{MLE} = \frac{S}{n}
\end{align}

\begin{align}
    f_{\Theta|X}(\theta|x) &\propto f_{X|\Theta}(x|\theta)\, f_\Theta(\theta)\nonumber\\
    &\propto \theta^{S-\frac{1}{2}}\,(1-\theta)^{n-S-\frac{1}{2}}
\end{align}
where $f_{\Theta|X}(\theta|x)$ is the posterior density.
\begin{align}
    \int_0^1 f_{\Theta|X}(\theta|x)\,d\theta = 1\nonumber\\
    \therefore f_{\Theta|X}(\theta|x) = \frac{\theta^{S-\frac{1}{2}}\,(1-\theta)^{n-S-\frac{1}{2}}}{B(S+\frac{1}{2},n-S+\frac{1}{2})}
\end{align}
where $B(x,y)$ is the beta function. From definition of beta function we get
\begin{align}
    B(x,y) &= \int_0^1 t^{x-1}\,(1-t)^{y-1}\,dt\nonumber\\
    &= \frac{x+y}{xy}\times \frac{1}{\comb{x+y}{x}}\label{eq1}
\end{align}
Let posterior mean be $E(\Theta)$
\begin{align}
    E(\Theta) &= \int_0^1 \theta\,f_{\Theta|X}(\theta|x)\,d\theta\nonumber\\
    &=\int_0^1 \frac{\theta^{S+\frac{1}{2}}\,(1-\theta)^{n-S-\frac{1}{2}}}{B(S+\frac{1}{2},n-S+\frac{1}{2})}\nonumber\\
    &=\frac{B(S+\frac{3}{2},n-S+\frac{1}{2})}{B(S+\frac{1}{2},n-S+\frac{1}{2})}\label{eq2}
\end{align}
Using \eqref{eq1} in \eqref{eq2} we get
\begin{align}
    E(\Theta) = \frac{S+\frac{1}{2}}{n+1}
\end{align}
Since $n>1$, $E(\Theta)$ exists.\\
For $E(\Theta)$ to be greater than MLE,
\begin{align}
    \frac{S+\frac{1}{2}}{n+1} &> \frac{S}{n}\nonumber\\
     n&>2\,S (S>0) \text{or } n<2\,S (S<0)\\
\end{align}
Since $n>1$, for $E(\Theta) >$  MLE we get $n>2\,S$.\\
$\therefore$ Option 2. and 4. are correct.\\
\end{document}