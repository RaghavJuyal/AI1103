\documentclass[journal,12pt,twocolumn]{IEEEtran}

\usepackage{setspace}
\usepackage{gensymb}
\singlespacing
\usepackage[cmex10]{amsmath}

\usepackage{amsthm}

\usepackage{mathrsfs}
\usepackage{txfonts}
\usepackage{stfloats}
\usepackage{bm}
\usepackage{cite}
\usepackage{cases}
\usepackage{subfig}

\usepackage{longtable}
\usepackage{multirow}

\usepackage{enumitem}
\usepackage{mathtools}
\usepackage{steinmetz}
\usepackage{tikz}
\usepackage{circuitikz}
\usepackage{verbatim}
\usepackage{tfrupee}
\usepackage[breaklinks=true]{hyperref}
\usepackage{graphicx}
\usepackage{tkz-euclide}


\usetikzlibrary{calc,math}
\usepackage{listings}
    \usepackage{color}                                            %%
    \usepackage{array}                                            %%
    \usepackage{longtable}                                        %%
    \usepackage{calc}                                             %%
    \usepackage{multirow}                                         %%
    \usepackage{hhline}                                           %%
    \usepackage{ifthen}                                           %%
    \usepackage{lscape}     
\usepackage{multicol}
\usepackage{chngcntr}

\DeclareMathOperator*{\Res}{Res}

\renewcommand\thesection{\arabic{section}}
\renewcommand\thesubsection{\thesection.\arabic{subsection}}
\renewcommand\thesubsubsection{\thesubsection.\arabic{subsubsection}}

\renewcommand\thesectiondis{\arabic{section}}
\renewcommand\thesubsectiondis{\thesectiondis.\arabic{subsection}}
\renewcommand\thesubsubsectiondis{\thesubsectiondis.\arabic{subsubsection}}



\hyphenation{op-tical net-works semi-conduc-tor}
\def\inputGnumericTable{}                                 %%

\lstset{
%language=C,
frame=single, 
breaklines=true,
columns=fullflexible
}
\begin{document}


\newtheorem{theorem}{Theorem}[section]
\newtheorem{problem}{Problem}
\newtheorem{proposition}{Proposition}[section]
\newtheorem{lemma}{Lemma}[section]
\newtheorem{corollary}[theorem]{Corollary}
\newtheorem{example}{Example}[section]
\newtheorem{definition}[problem]{Definition}

\newcommand{\BEQA}{\begin{eqnarray}}
\newcommand{\EEQA}{\end{eqnarray}}
\newcommand{\define}{\stackrel{\triangle}{=}}
\bibliographystyle{IEEEtran}
\raggedbottom
\setlength{\parindent}{0pt}
\providecommand{\mbf}{\mathbf}
\providecommand{\pr}[1]{\ensuremath{\Pr\left(#1\right)}}
\providecommand{\qfunc}[1]{\ensuremath{Q\left(#1\right)}}
\providecommand{\sbrak}[1]{\ensuremath{{}\left[#1\right]}}
\providecommand{\lsbrak}[1]{\ensuremath{{}\left[#1\right.}}
\providecommand{\rsbrak}[1]{\ensuremath{{}\left.#1\right]}}
\providecommand{\brak}[1]{\ensuremath{\left(#1\right)}}
\providecommand{\lbrak}[1]{\ensuremath{\left(#1\right.}}
\providecommand{\rbrak}[1]{\ensuremath{\left.#1\right)}}
\providecommand{\cbrak}[1]{\ensuremath{\left\{#1\right\}}}
\providecommand{\lcbrak}[1]{\ensuremath{\left\{#1\right.}}
\providecommand{\rcbrak}[1]{\ensuremath{\left.#1\right\}}}
\theoremstyle{remark}
\newtheorem{rem}{Remark}
\newcommand{\sgn}{\mathop{\mathrm{sgn}}}
\providecommand{\abs}[1]{\vert#1\vert}
\providecommand{\res}[1]{\Res\displaylimits_{#1}} 
\providecommand{\norm}[1]{\lVert#1\rVert}
%\providecommand{\norm}[1]{\lVert#1\rVert}
\providecommand{\mtx}[1]{\mathbf{#1}}
\providecommand{\mean}[1]{E[ #1 ]}
\providecommand{\fourier}{\overset{\mathcal{F}}{ \rightleftharpoons}}
%\providecommand{\hilbert}{\overset{\mathcal{H}}{ \rightleftharpoons}}
\providecommand{\system}{\overset{\mathcal{H}}{ \longleftrightarrow}}
	%\newcommand{\solution}[2]{\textbf{Solution:}{#1}}
\newcommand{\solution}{\noindent \textbf{Solution: }}
\newcommand{\cosec}{\,\text{cosec}\,}
\providecommand{\dec}[2]{\ensuremath{\overset{#1}{\underset{#2}{\gtrless}}}}
\newcommand*{\permcomb}[4][0mu]{{{}^{#3}\mkern#1#2_{#4}}}
\newcommand*{\perm}[1][-3mu]{\permcomb[#1]{P}}
\newcommand*{\comb}[1][-1mu]{\permcomb[#1]{C}}
\newcommand{\myvec}[1]{\ensuremath{\begin{pmatrix}#1\end{pmatrix}}}
\newcommand{\mydet}[1]{\ensuremath{\begin{vmatrix}#1\end{vmatrix}}}
\numberwithin{equation}{subsection}
\makeatletter
\@addtoreset{figure}{problem}
\makeatother
\let\StandardTheFigure\thefigure
\let\vec\mathbf
\renewcommand{\thefigure}{\theproblem}
\def\putbox#1#2#3{\makebox[0in][l]{\makebox[#1][l]{}\raisebox{\baselineskip}[0in][0in]{\raisebox{#2}[0in][0in]{#3}}}}
     \def\rightbox#1{\makebox[0in][r]{#1}}
     \def\centbox#1{\makebox[0in]{#1}}
     \def\topbox#1{\raisebox{-\baselineskip}[0in][0in]{#1}}
     \def\midbox#1{\raisebox{-0.5\baselineskip}[0in][0in]{#1}}
\vspace{3cm}
\title{Assignment 6}
\author{Raghav Juyal - EP20BTECH11018}
\maketitle
\newpage
\bigskip
\renewcommand{\thefigure}{\theenumi}
\renewcommand{\thetable}{\theenumi}
Download latex-tikz codes from 
%
\begin{lstlisting}
https://github.com/RaghavJuyal/AI1103/tree/main/Assignment6/Assignment6.tex
\end{lstlisting}
\section*{Statistics 2015 Paper I, Q.2 (a)}
Let $C$ be a circle of unit area with centre at origin and let $S$ be a square of unit area with $\brak{\frac{1}{2},\frac{1}{2}}$, $\,\brak{\frac{-1}{2},\frac{1}{2}}$, $\,\brak{\frac{-1}{2},\frac{-1}{2}}$ and $\,\brak{\frac{1}{2},\frac{-1}{2}}$ as the four vertices. If $X$ and $Y$ be two independent standard variates, show that
\begin{align*}
    \iint_C{\phi\brak{x}\,\phi\brak{y}\,dx\,dy} \geq \iint_S{\phi\brak{x}\,\phi\brak{y}\,dx\,dy}
\end{align*}
where $\phi\brak{.}$ is the pdf of $N\brak{0,1}$ distribution.
\section*{Solution}
\begin{definition}
PDF of normal distribution is given as
\begin{align}
    \phi_Z\brak{z} = N\brak{\mu,\sigma^2} = \frac{1}{\sqrt{2\pi\sigma^2}}e^\frac{-\brak{z-\mu}^2}{2\sigma^2}
\end{align}
\end{definition}

\begin{corollary}
\begin{align}
    \phi_X\brak{x} = \frac{1}{\sqrt{2\pi}}e^\frac{-x^2}{2}\label{eq1}\\
    \phi_Y\brak{y} = \frac{1}{\sqrt{2\pi}}e^\frac{-y^2}{2}\label{eq2}
\end{align}
\end{corollary}

\begin{proof}
Since $\phi\brak{.}$ is the pdf of $N\brak{0,1}$ distribution (given in question),
$\mu = 0$ and $\sigma^2 = 1$.
\begin{align*}
    \implies \phi_X\brak{x} = \frac{1}{\sqrt{2\pi}}e^\frac{-x^2}{2}\,\text{and}\,
    \phi_Y\brak{y} = \frac{1}{\sqrt{2\pi}}e^\frac{-y^2}{2}
\end{align*}
\end{proof}

\begin{lemma}
For circle $C$,
\begin{align}
    \iint_C{\phi\brak{x}\,\phi\brak{y}\,dx\,dy} = 1 - e^{\frac{-1}{2\pi}}  \label{eq9}
\end{align}
\end{lemma}

\begin{proof}
$C$ has unit area with centre at origin.
\begin{align}
    \implies &\pi\times r^2=1\\
    \implies &|r| = \frac{1}{\sqrt{\pi}}
\end{align}
For the area inside circle $C$ we get,
\begin{align}
    x^2+y^2\leq \frac{1}{\pi}
    \implies |y| \leq \sqrt{\frac{1}{\pi}-x^2}\label{eq3}
\end{align}
From \eqref{eq3} we get,
\begin{align}
    \iint_C{\phi\brak{x}\,\phi\brak{y}\,dx\,dy} = \nonumber\\
    \int\limits_{x=\frac{-1}{\sqrt{\pi}}}^{x=\frac{1}{\sqrt{\pi}}}\, \int\limits_{y=-\sqrt{\frac{1}{\pi}-x^2}}^{y=\sqrt{\frac{1}{\pi}-x^2}} \phi\brak{x}\,\phi\brak{y}\,dy\,dx \label{eq4}
\end{align}
Using \eqref{eq1} and \eqref{eq2} in \eqref{eq4} we get,
\begin{align}
    \int\limits_{x=\frac{-1}{\sqrt{\pi}}}^{x=\frac{1}{\sqrt{\pi}}} \int\limits_{y=-\sqrt{\frac{1}{\pi}-x^2}}^{y=\sqrt{\frac{1}{\pi}-x^2}} \frac{1}{2\pi}e^\frac{-\brak{x^2+y^2}}{2}\,dy\,dx
\end{align}
Converting it to polar coordinates we get,
\begin{align}
    &\int\limits_{r=0}^{r=\frac{1}{\sqrt{\pi}}} \int\limits_{\theta=0}^{\theta=2\pi} \frac{1}{2\pi}e^\frac{-r^2}{2}\,r\,d\theta\,dr\\
    &= \int_{0}^{\frac{1}{\sqrt{\pi}}}e^\frac{-r^2}{2}\,r\,dr\\
    &= 1 - e^{\frac{-1}{2\pi}}
\end{align}
\end{proof}

\begin{definition}
The $Q$ function is defined as
\begin{align}
    Q\brak{x} = \frac{1}{\sqrt{2\pi}}\int_x^\infty e^{\frac{-u^2}{2}}\, du\label{eq5}
\end{align}
\end{definition}

\begin{lemma}
\begin{align}
    \int_{-a}^{a} e^{\frac{-x^2}{2}}\, dx = \sqrt{2\pi}\,\brak{1-2Q\brak{a}}\label{eq6}
\end{align}
\end{lemma}

\begin{proof}
Since $e^{\frac{-x^2}{2}}$ is an even function,
\begin{align}
    &\int_{-a}^{a} e^{\frac{-x^2}{2}}\, dx = 2\int_{0}^{a} e^{\frac{-x^2}{2}}\, dx\\
    &\implies 2\int_{0}^{a} e^{\frac{-x^2}{2}}\, dx = 2 \brak{\int_{0}^{\infty} e^{\frac{-x^2}{2}}\, dx - \int_{a}^{\infty} e^{\frac{-x^2}{2}}\, dx} \nonumber\\
    &= 2\sqrt{2\pi} \times \frac{1}{\sqrt{2\pi}} \brak{\int_{0}^{\infty} e^{\frac{-x^2}{2}}\, dx - \int_{a}^{\infty} e^{\frac{-x^2}{2}}\, dx} \label{eq7}
\end{align}

Comparing \eqref{eq7} with \eqref{eq5} we get,
\begin{align}
    \int_{-a}^{a} e^{\frac{-x^2}{2}}\, dx &= 2\sqrt{2\pi}\brak{Q\brak{0} - Q\brak{a}}\\
    &= 2\sqrt{2\pi}\brak{\frac{1}{2} - Q\brak{a}}\\
    &= \sqrt{2\pi}\,\brak{1 - 2Q\brak{a}}
\end{align}
\end{proof}

\begin{lemma}
For square $S$,
\begin{align}
    \iint_S{\phi\brak{x}\,\phi\brak{y}\,dx\,dy} = \brak{1-2Q\brak{\frac{1}{2}}}^2 \label{eq10}
\end{align}
\end{lemma}

\begin{proof}
For square $S$ (given in question),
\begin{align}
    \frac{-1}{2}\leq x \leq \frac{1}{2}\\
    \frac{-1}{2}\leq y \leq \frac{1}{2}
\end{align}
From this we get,
\begin{align}
    \iint_S{\phi\brak{x}\,\phi\brak{y}\,dx\,dy} = \nonumber\\
    \int\limits_{x=\frac{-1}{2}}^{x=\frac{1}{2}} \int\limits_{y=\frac{-1}{2}}^{y=\frac{1}{2}} \phi\brak{x}\,\phi\brak{y}\,dy\,dx \label{eq8}
\end{align}
Using \eqref{eq1} and \eqref{eq2} in \eqref{eq8} we get,
\begin{align}
    \int\limits_{x=\frac{-1}{2}}^{x=\frac{1}{2}} \int\limits_{y=\frac{-1}{2}}^{y=\frac{1}{2}}\frac{1}{2\pi}e^\frac{-\brak{x^2+y^2}}{2}\,dy\,dx
\end{align}
Using \eqref{eq6} twice we get,
\begin{align}
    \int\limits_{x=\frac{-1}{2}}^{x=\frac{1}{2}} \int\limits_{y=\frac{-1}{2}}^{y=\frac{1}{2}}\frac{1}{2\pi}e^\frac{-\brak{x^2+y^2}}{2}\,dy\,dx &= \brak{1-2Q\brak{\frac{1}{2}}}^2
\end{align}
\end{proof}
\begin{solution}
Calculating the values of \eqref{eq9} and \eqref{eq10} we get,
\begin{align}
    1 - e^{\frac{-1}{2\pi}} &= 0.147136\\
    \brak{1-2Q\brak{\frac{1}{2}}}^2 &= 0.146631
\end{align}
This proves that 
\begin{align}
    &1 - e^{\frac{-1}{2\pi}} \geq \brak{1-2Q\brak{\frac{1}{2}}}^2\\
    \implies  &\iint_C{\phi\brak{x}\,\phi\brak{y}\,dx\,dy} \geq \iint_S{\phi\brak{x}\,\phi\brak{y}\,dx\,dy}
\end{align}
\end{solution}

\end{document}